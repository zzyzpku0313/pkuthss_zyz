\chapter{问题及思考}
\label{chap:problem}

本模板存在一些问题,这些问题对论文写作可能影响没有那么大,但终究不够完美。本人通过一些不那么“美”的方式进行了解决,期待使用者找到更好的解决方案。

\section{封面问题}

封面存在两个问题,其一是模板默认标题有两行下划线,不论标题几行,下划线都是两行。其二是自2024年始,封面中加入了“学术学位”和“专业学位”的选项,该模板并没有。

推荐的解决方法:封面用学校提供的word文档编辑,转为pdf格式,然后用pdf编辑软件替换本模板的封面。或者用pkuthss v1.9.4模板,已解决该问题。

\section{目录问题}

本模板没有找到合适的将“目录”放在目录里的方法,因此目录里没有“目录”一项。另外,为了将“摘要”和“ABSTRACT”放入目录,在\texttt{abs.tex}中使用了 \texttt{\string\addcontentsline\{toc\}\{chapter\}\{摘要\}} 命令,使得编译后的pdf文档在书签侧栏可能出现两个“摘要”和“ABSTRACT”,手动删去一个即可。

此外,本模板也提供了插入图、表索引的选择,在\texttt{main.tex}中可选。请注意,图表索引是将图表的\texttt{\string\caption}中的文字全部加入目录,因此图注中有较多说明文字时(比如红线代表了xxx,蓝色三角形代表了xxx),不适宜插入图表索引。

\section{字体问题}

本模板是基于pkuthss v1.9.2开发的,该模板默认字体是\texttt{fandol},并非宋体。其优点是免费使用,overleaf支持,且和宋体几乎无差;其缺点是字体库较少,对一些生僻字可能无法显示。比如本人姓名中的“吉吉”(zhe2),是无法显示的。由于本人的论文中没有用生僻字,所以只有在封面和个人简历中可能出现姓名显示不完全。对此,本人的做法是在编译后的pdf文档中修改。

本人测试发现,在windows环境下,配置好本地环境,使用VSCode编译,设置\texttt{fontset=windows},生僻字问题可以解决,且字体也和宋体几乎无差。

目前最新版的pkuthss v1.9.4支持了宋体,有兴趣的同学可以使用该版本试一试。但该模板代码风格和v1.9.2版本有些区别,并不能直接移植过去,使用者可对比选择。

\section{段落间距问题}

使用word排版论文时,段落间距一般和行间距是一样的。但使用本模板在排版时,有时会遇到段落间距比行间距更大的情况,使段间空行比较突出。比如一页的最后出现章节标题之时,为了使章节标题不会出现在本页最后一行,该模板在排版时会倾向于将本页其他段落段间距加大,让章节标题排在下一页。又或者,正常排版时,下一页的第一行是一个公式,那么该模板会让本页的段间距加大,从而让一些文字被“挤”到下一页,避免下页第一行是公式。

这个问题遇到了自然明白,一般论文写作不会出现大篇幅的段间距过大,即便有几页如此,也并不显得不美观,因此本人并未对此进行任何修正。在致谢部分(见\texttt{ack.tex}),有添加\texttt{\string\setlength\{\string\parskip\}\{0pt\}}的代码禁止段间距离拉长,使用者可以注释掉这行代码查看该问题的效果。

\section{论文补充页面的插入}

提交最终版论文一般需要插入版权声明、原创性声明等等页面。在本模板中,插入“版权声明”(见\texttt{copy.tex})和“北京大学学位论文原创性声明和使用授权说明”页面(见\texttt{origin.tex})是通过插入图片的方式实现的。然而,从个人门户下载的pdf文件并不能直接插入,可以将这些文件先用Adobe Illustrator打开,再另存为pdf文件,使其变为矢量图,就可以插入了。当然,也可以在写作时不添加这些页面,最终版论文插入这些文件的扫描件即可。

\section{其他问题}

北大未名bbs论坛反映了pkuthss v1.9.2的一些其他问题,比如一些数学符号字体不正确等,本人并未遇到。如有任何问题可在bbs上提问和讨论。