\chapter*{个人简历、在学期间科研成果与荣誉}
\markboth{个人简历、在学期间科研成果与荣誉}{}
\phantomsection
\addcontentsline{toc}{chapter}{个人简历、在学期间科研成果与荣誉}

\subsection*{\bfseries 个人简历}

何年何月出生何地

何年何月何地本科

何年何月何地博士,导师

\subsection*{\bfseries 发表论文}

\setlength{\parskip}{6pt}
\renewcommand\labelenumi{[\theenumi]}

\begin{enumerate}
  \item \textbf{Mr. ZYZ}, others. (2024). Paper name. \textit{Journal name, 1}, 001-010. %\url{https://doi.org/xxxx}
  \item \textbf{Mr. ZYZ}, others. (2024). Paper name. \textit{Journal name, 1}, 001-010. %\url{https://doi.org/xxxx}
\end{enumerate}

\subsection*{\bfseries 参加学术会议}
\setlength{\parskip}{6pt}
\renewcommand\labelenumi{[\theenumi]}

\begin{enumerate}

    \item \textbf{Mr. ZYZ}, others. Presentation name. Dec. 2023, \textit{AGU Fall Meeting 2023}, San Francisco, USA, Oral presentation.

\end{enumerate}

\subsection*{\bfseries 获得荣誉}
\setlength{\parskip}{6pt}
\renewcommand\labelenumi{[\theenumi]}

\begin{enumerate}

  \item 奖项1,2019年12月,2021年12月
  \item 奖项2,2023年12月
\end{enumerate}

\subsection*{\bfseries 其他项目}
\setlength{\parskip}{6pt}
\renewcommand\labelenumi{[\theenumi]}

可以根据需要添加,例如科研项目、野外工作、专利、软件、学术服务(比如审稿人)等等。

