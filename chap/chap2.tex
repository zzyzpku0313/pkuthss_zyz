\chapter{参考文献排版}
\label{chap:reference}

参考文献的引用和排版可以采用“顺序编码制”或“著者-出版年制”,本模板均通过符合GB/T 7714-2015 标准的biblatex 参考文献宏包(以下简称7714宏包)实现,在\texttt{main.tex}文件中选择其中一种方式即可。“顺序编码制”排版规则简单,使用时需要调整的参数很少。使用“著者-出版年制”时,需要调整的参数较多。以下给出本模板对参考文献排版的设计,以及示例、问题和思考。

\section{本模板的设计}

本人写作时曾惊讶于7714宏包的强大,使用者请一定查阅7714宏包说明文档,即本模板提供的\texttt{biblatex-gb7714-2015.pdf}文件,以了解各种用法。该宏包的相关信息也可网上搜索得到。

本模板对“著者-出版年制”选项设置为\texttt{\string\usepackage[backend = biber, style = gb7714-2015ay, gbnamefmt=familyahead, gbtype=false, maxcitenames=2, mincitenames=1, uniquelist=false, gbcitelabel=parensqj]\{biblatex\}}。主要选项意义如下:

\begin{itemize}
    \item \texttt{gbnamefmt=familyahead}:排版时姓在前名在后,类似于APA的样式;
    \item \texttt{gbtype=false}:不输出文献类型标识符。
    \item \texttt{maxcitenames=2,mincitenames=1}:超过\texttt{maxcitenames}个作者的文献,只输出\texttt{mincitenames}个作者,并添加“et al.”或“等”。
    \item \texttt{uniquelist=false}:\texttt{uniquelist}用于正文中引用(标注)标签的作者列表控制(目的是消除歧义)。这里设置为\texttt{faluse}表示仅用一个作者作为文中的标注标签,如果遇到同姓作者同一年的文献,通过在年份后添加字母以区分。
    \item \texttt{gbcitelabel=parensqj}:标注标签由全角圆括号包围。由于引用时会出现大量括号,该设置的目的是使正文中不会出现英文、中文括号混用的现象。
    \item 增加了中文文献用“和”、英文文献用“and”连接人名的设置。
\end{itemize}

\section{示例}
\label{sec:ref_example}

我们以北大论文word模板中的一段文字进行参考文献的示例。修改\texttt{main.tex}中的参考文献排版方式,可以对比“著者-出版年制”和“顺序编码制”两种方式的参考文献排版方式。

\texttt{环境中黑炭(black carbon)气溶胶的主要来源包括各种化石燃料和生物质燃料的不完全燃烧过程\cite{Penner1993,Bond2004},这些不完全燃烧在自然界和人类活动中都会发生,因此,环境中黑炭气溶胶的来源十分广泛。对当今大气环境中的黑炭,其主要来源是人类相关的燃料燃烧活动\cite{段凤魁_2007},此外,一些自然过程也会产生黑炭,如森林火灾、草原火灾等。根据过去的排放清单研究,大气环境中黑炭气溶胶的来源主要包括:1)有机燃料的燃烧,主要包括能源行业、工业部门、交通运输行业、居民生活中煤、石油、天然气和各种生物质燃料的使用。通常而言,燃烧效率越高,产生的黑炭气溶胶的量越低;2)工业炼焦,主要包括炼焦过程中的炼制过程、焦炉加热系统以及焦炉煤气的泄漏等等;3)工业制砖,主要包括制砖过程中物料破碎输送、坯体人工干燥和烧成工段等过程;4)垃圾焚烧,包括生活垃圾和工业废料的燃烧过程;5)天然火灾和野外农业废弃物燃烧,包括森林、草原火灾和秸秆的燃烧。目前大部分研究表明,民用取暖和做饭过程中的燃料燃烧和城市柴油车是黑炭气溶胶大气排放量最大的源\cite{Streets2001,Streets2003,Bond2004,Bond2006,Cao2006,Klimont2009,Zhang2009,Lu2011}。}

\section{参考文献引用方式}

论文引用参考文献一般有两种方式,一种是仅注释,使用\texttt{\string\cite},效果是“这个问题一般是这样的\cite{Yao_2015}”。一种是作者作为主语,使用\texttt{\string\textcite},即“\textcite{Yao_2015}表明,这个问题是这样的”。此外,\texttt{\string\citep}命令可以实现参考文献的括号中加入前缀和后缀,例如,“这个问题已被解决\citep[见 \ref{sec:ref_example} 节;][]{Bond2006}”。

然而,只要文章中出现作者姓名,就会引发一些问题,比如文献多了该怎么排序。北大word模板对这些问题并没有做明确的规定,本人及同学们写作过程中只能参考该模板,选择认为比较规范且言之有理的方式。这里给出两个问题的思考,本人的处理方式并非官方解答。

1.有两个作者A和B时,是引用A and B,还是引用A et al.?

北大给出的模板未对此做明确规定,但其给出的示例是引用A et al.,对比word模板和本文 \ref{sec:ref_example} 节引用的\texttt{Bond2006}这篇文献即可发现。但一方面,这种引用方式并不是明文规定,另一方面,本专业投稿时一般的使用习惯(以AGU期刊为例)是引用两个作者,三个及以上的作者才会用“第一作者 et al.”。所以本模板通过设置7714宏包调用时的参数,实现了作者是两个人时,引用完全,超过两个人就只引“第一作者 et al.”。

2.如果引用多篇参考文献,是按照人名首字母的先后顺序排列,还是按照年份顺序排列?

对这个问题,北大word模板给出的示例是按照时间顺序排序,如果有几篇文章引用时作者相同,就按其最早的一篇时间排序。本模板给出的排序方法是按人名首字母优先排序(也是参照了AGU期刊的习惯)。另外,对第一作者相同、但作者列表并不完全相同的几篇文献,可能引用后都是A et al.,这时也会先判断“et al.”所省略的作者姓名的顺序,而并非仅按照时间顺序排列。对此,本人认为这不一定是最好的排序方法,只是在未有明文规定情况下一种言之有理的排序方案。

以上这些问题都可以通过探索7714宏包的参数设置自行调整。

\section{参考文献列表排版}

7714宏包给出了很多可用的文献列表格式,比如是否标注文献类型(期刊用[J]、专著用[M]等)、人名缩写加不加点(Zhang S. 还是 Zhang S),使用者可以自行探索。这里介绍两个可能出现的问题。

1.汉语文献多音字问题。

本人写作时,发现自动排列参考文献时,可能出现由于多音字造成的参考文献顺序的误排。例如,“曾”作为姓是“zeng1”,但可能作为“ceng2”被排在列表前方;沈“shen3”可能被作为“chen2”排序。7714宏包给出了这个问题的解决方案(请参照说明文档),本模板采用的是对文献表中的\textbf{全部}中文文献都设置key域(全部中文文献都要设置,否则会失效),即标注著者姓名的拼音。例如,在文件\texttt{ref.bib}中,\texttt{段凤魁2007}这篇文献的作者是\texttt{author = \{段凤魁 and 贺克斌 and 刘咸德 and 董树屏 and 杨复沫\}},相应的key域就是\texttt{key = \{duan4feng4kui2 and he4ke4bin1 and liu2xian2de2 and dong3shu4ping2 and yang2fu4mo4\}}。

给BibTex文件中的全部中文文献添加key域是一个繁琐的工作,手动添加耗时且易错。7714宏包的说明文档给出了自动添加的方案。本人的做法是使用Python自动读取BibTex文件的每条参考文献,然后再自动生成人名拼音,并在每一条中文文献中添加key域,最后输出新的BibTex文件。这个过程可以交给AI自动生成代码,再自己修改。AI对BibTex这种格式固定的文件的处理还是比较准确的。

2.BibTex文件的手动修改。

使用LaTex可能会陷入一个误区,即所有问题可以通过代码解决。尤其是LaTex对参考文献排版有诸多便利之处,往往使人忽略一些简单的问题。本人建议在论文定稿前,对BibTex文件中的每条参考文献都人工检查并手动完善一遍,确保格式统一、没有排不出来的字符、排版时没有问题。

举个例子,本人搜集文献信息一般是从期刊官网导出文献信息,再导入Endnote中。然而,同一作者在不同期刊官网导出的姓名可能不同,比如张一三发表的文献1在A期刊导出为\texttt{Zhang, Yi San},文献2在B期刊导出为\texttt{Zhang, Y. S.},文献3在C期刊可能导出为\texttt{Zhang Y.},文献4在D期刊导出为\texttt{Zhang Yi-San}。于是,使用Endnote生成BibTex文件时,文献1$\sim$4中的张一三会被识别为不同的人,导致引用和排序时出现错误。因此,一定要手动检查所有参考文献,确保\texttt{author}一栏人名格式是统一的。
