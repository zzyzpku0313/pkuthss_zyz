\chapter{引言}
\label{chap:introduction}

对论文写作而言,LaTex在公式图表的插入及引用、参考文献的引用和排版等方面相较word更为便捷,然而北京大学博士研究生学位论文的官方模板是以word形式发布的,未有官方的LaTex模板。本人进行论文写作时,找到一些符合北大博士论文要求的LaTex模板,主要是pkuthss系列和北京大学工学院的模板。

本模板主要以pkuthss v1.9.2为基础,也融合了工学院模板的一些优点,主要特点是对“著者-出版年制”的参考文献引用方式做了更具体的设置,较pkuthss v1.9.2原始模板更规范。本模板更像是本人使用pkuthss的经验总结,可以作为使用者(尤其是地物专业的同学)的一种参考,也可以作为LaTex初学者的一个样本。模板中的一些缺点也将在后续章节说明。LaTex有许多使用技巧,使用者可根据需求搜索攻略,期待后来人能将该模板不断完善。

\section{参考模板}

提供三个北大博士学位论文的LaTex模板。

\begin{enumerate}[itemindent=0.3em]     % 用enumerate环境可能需要控制缩进量
    \item pkuthss v1.9.2:\url{https://github.com/iofu728/pkuthss}。
    \item pkuthss v1.9.4:\url{https://github.com/zhiyunyao/pkuthss/tree/lite}。该模板发布于2024年4月,推荐使用这一最新模板,将本模板作为参考即可。
    \item 工学院模板:\url{https://www.coe.pku.edu.cn/service/biyedb/11187.html}。本模板中研究生成果页参考了该模板。
\end{enumerate}

\section{编译要求}

本模板仅支持UTF-8文件编码和XeLaTex编译。请确保所有文件为UTF-8编码,编译时请修改编译器为XeLaTex。

\section{环境配置}

\subsection{在VSCode使用}

在本地环境下进行写作,推荐使用VSCode。

环境配置:安装TeX Live(可能需要8G左右的空间),在VSCode中配置LaTex Workshop扩展。相关方法网上已有许多。

需要注意的是,如果已安装了其他Tex系统(比如CTEX),可能与TeX Live冲突,VSCode编译时可能优先选择其他系统。使用者可以先试一下能否编译成功,再决定是否安装Tex Live。如安装Tex Live后仍编译不成功,需要卸载有冲突的Tex系统。

使用模板:在GitHub上(\url{https://github.com/zzyzpku0313/pkuthss_zyz})下载该项目的zip压缩包,解压后在VSCode中打开\texttt{main.tex},用XeLaTex编译即可。

\subsection{在overleaf使用}

如不想在本地进行环境配置,也可以选择overleaf进行在线写作,只需将本模板包的所有文件copy到新建的项目里,或将GitHub下载的zip包上传到oveleaf中,用XeLaTex编译\texttt{main.tex}即可。

需要注意的是,overleaf对编译超过20 s的项目要收费,必须开通会员才可使用。截止2024年6月,学生会员的价格是9 $\$$ /月。

本模板项目较小,编译时间暂不到20 s,如编译一篇学位论文则必然超过20 s。

\section{模板结构}

本模板的文件结构不再详细说明,依据文件名及内容均可一目了然。一些文件的示例(比如\texttt{deno.tex})来自已归档的博士论文。

本说明文档行文结构如下:

第\ref{chap:introduction}章将介绍几种北大学位论文LaTex模板,并介绍本模板的使用方法。

第\ref{chap:reference}章将介绍本模板中参考文献的引用和排版方式。

第\ref{chap:fig_tab}章将介绍图表的插入、引用方式及使用中的问题和技巧。

第\ref{chap:equation}章将介绍公式的插入、引用方式及使用中的问题和技巧。

第\ref{chap:problem}章将介绍本模板存在的一些问题。

第\ref{chap:conclusion}章给出结论和展望的模板。
